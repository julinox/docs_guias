% Main
\documentclass{article}

% Packages
\usepackage[spanish]{babel}

% Headers
\title{Funcion Hash}
\author{Juli No}

%MACROS
\newcommand{\SP}{\,\,}

% Document
\begin{document}
\maketitle
\pagebreak

\section{Funcion Hash}
Un hash es una funcion que genera una salida de tamaño constante independientemente
de la entrada recibida, es decir, sea h la funcion Hash y sea
m una entrada tenemos:
$$
H = h(m)
$$

El tamaño de la salida es la cantidad de bits que especifique la funcion hash,
ejemplo: SHA-256 tiene 256 bits, es decir que genera una salida cuya tamaño son 256
bits ($256/8 = 32$ caracteres si hablamos de ascii)\\

Una caracteristica muy importante que tienen o deben tener la funcion Hash es que
no es invertible, es decir, dado $H = h(m)$ no puede encontrarse o es extremadamente
dificil de encontrar $h^{-1}(H) = m$

\subsection{Colisiones}
Se conoce como colision al hecho de generar 2 hashes (salidas) iguales a partir de 2 mensajes distintos,
es decir:
$$
H = h(m_{1}), \SP H = h(m_{2}), \SP m1 \neq m2
$$

\subsection{Probabilidad de una Colision}
Supongamos una funcion hash de 4 bits de tamaño de salida, los posibles estados
que tendremos sera: \textbf{0000, 00001, 0010, 0011, 0100, 01001, 0110, 0111, 1000,
1001, 1010, 1011, 1100, 1101, 1110, 1111}. Tenemos 16 estados posibles, dado que cada
estado tiene la misma probabilidad de ocurrencia tenemos que la probabilidad del estado
e es:
$$
P_{e} = \frac{1}{2^4} = \frac{1}{16}
$$

De manera mas general, si el algoritmo de hashing genera una salida de n bits
la probabilidad de una colision viene dada por:
$$
P_{c} = \frac{1}{2^n}
$$

\subsection{Propiedades}
\begin{itemize}
    \item \textbf{Facilidad de calculo:} $h(m) = H$ rapido de calcular
    \item \textbf{Unidireccionalidad:} Dado $h(m) = H$ es muy dificil encontrar m a partir de H
    \item \textbf{Compresion:} Dado m la salida H debe ser siempre el mismo tamaño
    \item \textbf{Difusion:} Un cambio en un bit para la entrada m generan $2 \SP H_{s}$ completamente distintos
    \item \textbf{Resistencia simple a colisiones:} Sea $h(m_{1}) = H$, encontrar $H = h(m_{2})$ con $m_{1} \neq m_{2}$ debe ser
    computacionalmente dificil
    \item \textbf{Resistencia fuerte a colisiones:} Debe ser computacionalmente dificil encontrar 2 mensajes $(M_{1}, M_{2})$ al azar
    que produzca el mismo H. Por la paradoja del cumpleaños la probabilidad de ocurrencia
    de esto es $1/2^{n/2}$
\end{itemize}
\end{document}