\documentclass{article}
\usepackage[spanish]{babel}
\usepackage{amsmath, amsthm, amsfonts}

\title{Matematicas Discretas y Criptografia}
\author{Julino}
\begin{document}
\maketitle
\newtheorem{theorem}{Teorema}[section]

%MACROS
\renewcommand \qedsymbol{}
\renewcommand{\thefootnote}{\fnsymbol{footnote}}
\newcommand{\SP}{\,\,}

\pagebreak
\section{Aritmetica Modular}
Todo numero entero 'p' $\in \mathbb{Z}$ puede expresarse de la siguiente manera:
\begin{equation}
    \label{divisibilidad}
    \begin{split}
        p &= nq + r \SP\SP \vert \SP n,q \in \mathbb{Z} \SP y \SP r \geq 0\\
        r &= p - nq
    \end{split}
\end{equation}

Eso significa que si R = 0, tanto n como q dividen a p, sin embargo nos enfocaremos
en n para trabajar en la aritmetica modular\\

Definimos entonces la aritmetica modular como el resto que se obtiene al dividir un
numero 'A' entre un modulo 'N'.
$$A\mod{N} = R \implies A = Nq + R\SP\SP \vert \SP\SP 0 \le \SP  R < N$$
\textbf{Ejemplo:} $16 \mod{5} = 1$\\

Se puede calcular el modulo N de un numero negativo, sin embargo en la Criptografia
nos interesa trabajar solamente con numeros positivos. Aun asi esta es la manera de
trabajar con numeros negativos:

$$A\mod{N} = (A + Nk) \mod{N}$$
\textbf{Ejemplo:} $-5 \mod{3} = (-5 + 3\times2) \mod{3} = 1\mod{3} = 1$\\
Nota: Siempre que $A < N, \SP A \mod{N} = A$

\subsection{Congruencia} \label{congruencia}
Definimos la conguencia de dos numeros enteros A y B con el modulo N ($A \equiv B \mod N$) si ambos
numeros generan el mismo resto R ($A = Nq + R$ y $B = Nq^\prime + R$)

\begin{theorem}
    $$ A \equiv B \mod N \implies N / (a-b)$$
\end{theorem}

\begin{proof}
    Usando \ref{congruencia} tenemos que $a-b = Nq + r -Nq^{\prime} - R = N(q-q^{\prime})$. Pero $q-q^\prime \in \mathbb{Z} \implies a-b = Nq^{\prime \prime} \implies N/(a-b)$
\end{proof}

\section{Numeros Primos y Compuestos}
Las operaciones modulares en Criptografia se realizan dentro de un modulo de cifra, cuyo
valor puede ser un numero primo o compuesto.\\

Un numero primo es aquel numero que solo es divisible por 1 y por el mismo,
es decir, sea P primo $P \in \mathbb{N}, \SP P > 1$ y los unicos divisores de P son 1 y P\\

Un numero compuesto es un numero natural no primo. Sea N un numero compuesto,
$N \in \mathbb{N}, \SP N > 1 \SP y \SP \exists n \in \mathbb{N}, \SP n \notin \{1,N\}
\SP \vert \SP n/N$

\subsection{Cardinalidad de los Numeros Primos}
Existen infinitos primos pero la cantidad de primos que existe en el intervalo [2, X]
viene dada por el teorema de los numeros primos:
\begin{theorem}
    $$|[2,x]| \approx \pi(x) \approx \frac{x}{ln(x)}$$
\end{theorem}

\begin{proof}
    La demostracion es sencilla y se deja como ejercicio al lector
\end{proof}

\textbf{Ejemplo:} ¿Cuantos primos hay de $2^{1024}$ bits?\\\\
Es decir, queremos calcular cuantos primos existen que se representen con 1024 bits.
Para ello calculamos cuantos primos en total hay en $2^{1024}$ y le restamos los
primos que necesiten menos de 1024 bits para representarlos. Sea $P_{1024}$ la cantidad
de primos de 1024 bits:
$$P_{1024} \approx \pi{(2^{1024})} - \pi{(2^{1023})} \approx 1,26 \times 10^{305}$$
En 'cristiano': ¡Es una cantidad inmesamente grande!

\subsection{Primo Seguro}
Se define primo seguro al numero primo P que satisface la siguiente condicion:
$P = 2P^{'} + 1$, donde $P^{'}$ es primo tambien. \textbf{Ejemplo:} $23 = 2 \times 11 + 1$. Claramente no
todos los $2P^{'} + 1$ son numeros primos, como ejemplo tenemos
al numero \textbf{15}

\subsection{Primo Relativo}
\label{primo_relativo}
Es un concepto comparativo, se dice que 2 numeros $a, b \in \mathbb{P}$ (donde $\mathbb{P}$
es el conjunto de los primos) son coprimos o primos relativos $\iff mcd(a,b) = 1$. Es
decir no tienen factores en comun. 

\subsection{Numero Compuesto}
Un numero compuesto es cualquier numero que tiene mas de 2 divisores, es decir se puede dividir por 1, por
el mismo y por algun otro conjunto de factores primos. Todo numero no primo es compuesto. En este caso
solo nos interesan los numeros compuestos del tipo $p \times q$, con p y q primos.\\
\textbf{Ejemplo:} $3 \times 5 \times 7$ es un numero compuesto del tipo que no nos interesa

\section{Conjunto de Restos}
El conjunto de resto es el conjunto que contiene todos los posibles restos de un
modulo N. Existen 2 tipos, el completo y el reducido. Es de hacer notar que este
ultimo es muy importante en la Criptografia asimetrica.

\subsection{Conjunto Completo de Restos}
Es el conjunto que contiene a todos los restos del modulo n:
$$CCR = \{0,1,2,...,n-2,n-1\}$$

\subsection{Conjunto Reducido de Restos}
Es el conjunto que contiene todos los restos que son primos relativos(\ref{primo_relativo}) con n.
$$CRR=\{1,n_{1},n_{2},...n_{i}\} \SP / \SP n_{i} < n, \SP mcd(n_{i}, n) = 1$$
Si n es primo entonces $CRR = \{1,2,3,...,n-2,n-1\}$. Vemos que el 0 se descarta
ya que no es solucion ($mcd(0, n) \neq 1$). Por supuesto que $CRR \subset CCR$.\\\\
Ejemplo: n = 8, $CCR = \{0,1,2,3,4,5,6,7\}, \SP CCR = \{1,3,5,7\}$

\subsection{Funcion de Euler($\phi(n)$)}
\label{fn_euler}
Gracias al gran matematico Leonhard Euler tenemos la funcion que lleva su nombre
que nos permite calcular el cardinal del CRR ($\phi(n) = \vert CRR \vert$). Este numero
es muy importante y servira para conseguir el inverso de un modulo n, sobre
eso hablaremos mas adelante.\\\\
Existen 4 casos donde $\phi(n)$ funciona:
\begin{itemize}
    \item n es primo
    \item $n = p^{k}$, p es primo y $k \in \mathbb{Z}$
    \item $n = p \times q$, p y q primos
    \item $n = p_{1}^{e_{1}} \times p_{2}^{e_{2}} \times ... \times p_{n}^{e_{n}} = \displaystyle\prod_{i=1}^{n}p_{i}^{e_{i}}$
\end{itemize}

Estudiaremos los unicos 2 casos que nos importa para los objetivos de este documento (que es la Criptografia)

\subsubsection{Caso n primo}
Si n es primo entonces $\phi(n) = \vert \{1,2,,...,n-2,n-1\} \vert = \vert CRR \vert = n-1$\\
\textbf{Ejemplo:} Sea n=7, $\phi(7)= 7-1 = 6$. $CRR_{7}=\{1,2,3,4,5,6\} \implies \vert CRR \vert = 6$ 

\subsubsection{Caso n 'compuesto'}
Si p y q son primos entonces $\phi(n) = \phi(p \times q) = \phi(p) \times \phi(q) = (p-1) \times (q-1)$
\begin{proof}
    Queda pendiente
\end{proof}
\textbf{Ejemplo:}0 Sea n = 15($5 \times 3$), $\phi(15) = (p-1) \times (q-1) = (5-1) \times (3-1) = 8$.
$CRR_{15} = \{1,2,4,7,8,11,13,14\}$

\section{Inversos}
Esta es sin duda la parte mas importante ya que en Criptografia (al menos en la asimetrica)
el inverso es lo que nos permite deshacer o revertir una operacion, si ciframos con un numero
C el inverso de C (inv(c)) nos permitira descifrar, y si ciframos con inv(C) entonces C nos
permitira descifrar, es decir, uno deshace lo que hizo el otro.\\

Existen 3 tipos de inversos que veremos: Aditivo, XOR y Multiplicativo, siendo este ultimo
el que se usa en la Criptografia asimetrica

\subsection{Inverso Aditivo}
Cifrado y Descifrado, donde 'K' es la clave:
\begin{align}
    \notag c_{i} &= m_{i} + K \mod n\\
    \label{desc_plus}
    m_{i} &= c_{i} + inv_{+}(K,n) \mod n
\end{align}

En el caso aditivo es muy sencillo encontrar el inverso:
$$inv_{+}(K,n) = n-k, \SP n > K$$
Entonces el inverso aditivo (\ref{desc_plus}) queda de la siguiente manera:
$$m_{i} = c_{i} + (n-K) \mod n$$

\subsection{Inverso XOR}
Es la misma operacion para cifrar y describrar (K es la clave):
\begin{align}
    \notag Cifrado:& \SP C = M \oplus  K\\
    \notag Descifrado:& \SP M = C \oplus K
\end{align}

\subsection{Inverso Multiplicativo}
El inverso multiplicativo de un \textbf{resto} k (logicamente $k<n$), no siempre existe.

\begin{itemize}
    \item $inv_{\times}(k,n) \equiv 1 \mod n \iff mcd(k,n) = 1$ (i.e $k, n$ coprimos)
    \item $k \times inv_{\times}(k,n) \mod n = 1 \implies mcd(k \times i_{k}, n) = 1 \SP \vert \SP i_{k} = inv_{\times}(k, n)$
    \item Si n es primo entonces k tiene inverso
    \item Si n es compuesto y $mcd(k, n) \neq 1$ entonces k no tiene
    inverso.
\end{itemize}

\begin{align}
    \notag Cifrado:& \SP C = k \times m_{i} \mod n\\
    \notag Descifrado:& \SP M = c_{i} \times inv_{\times}(k,n) \mod n
\end{align}

Pregunta: $inv_{\times}(k,n) < n$? No lo se pero si existe y es menor a n entonces el inverso
es unico dentro de los restos de n.

\section{Calculo del Inverso Multiplicativo ($inv_{\times}(k,n)$)}
Existen varias maneras para calcular el inverso multiplicativo:

\subsection{Fuerza Bruta}
Se prueban todos los restos de n. Si n es primo entonces la cantidad de restos a probar es
n-1 (excluyendo el 1). Si n es compuesto entonces solo se prueba con los restos que sean
coprimos con el modulo ($mcd(i^{'}, n) = 1$)

\subsection{Pequeño Teorema de Fermat}
\begin{equation}
    \label{ptf}
    Sea \SP n \SP primo \SP y \SP a > 0 \SP coprimo \SP con \SP n, \SP a^{n-1} \equiv 1 \mod n
\end{equation}

Sea a un resto dentro del modulo n, usando \ref{ptf} tenemos que $a^{n-1} \mod n = 1$, por otra parte tenemos que
$a \times x \mod n = 1$ (x es $inv_{\times}(a,n)$). Entonces tenemos lo siguiente:

\begin{align}
    \notag a^{n-1} \mod n &= a \times x \mod n\\
    \notag a ^{n-2} \mod n &= x \mod n\\
    \notag x &= a^{n-2} \mod n \SP \footnotemark
\end{align}
\footnotetext{Dado que $x < n \implies x \mod n = x$}

Usando la funcion de Euler (\ref{fn_euler}) podemos reescribir la ecuadion:
\begin{align}
    \notag \phi(n) &= n-1\\
    \notag x &= a^{\phi(n) -1}
\end{align}

Cuando n es compuesto tengo que calcular el $\phi(n)$. Si n es muy grande el calculo del inverso
usando al pequeño fermat sigue siendo muy costoso por lo que este metodo no sirve para la
Criptografia asimetrica donde los n son enormes.

\subsection{Algoritmo Extendido de Euclides (AEE)}
Si bien es mejor que el pequeño fermat este metodo sigue siendo lento para un n grande. No
obstante aca esta el algoritmo:

\begin{verbatim}
    // Esto es psecodigo
    // Linea blanca intencional
    // Linea blanca intencional
    // Linea blanca intencional
    // Linea blanca intencional
    // Linea blanca intencional
    func AEE(a, n)
        (g[0],g[1],u[0],u[1],v[0],v[1],i) = (n,a,1,0,0,1,1)

        while g[i] != 0
            y[i+1] = g[i-1]/g[i] // Parte entera
            g[i+1] = g[i-1]-(y[i+1]*g[i])
            u[i+1] = u[i-1]-(y[i+1]*u[i])
            v[i+1] = v[i-1]-(y[i+1]*v[i])
            i = i+1
        endwhile

        if v[i-1] < 0
            v[i-1] = v[i-1] + b
        endif

        return v[i-1]
    endfunc
\end{verbatim}

\subsection{Algoritmo Exponenciacion Rapida (AER)}
Este es el que se usa en Criptografia asimetrica y queda como tarea queda pendiente de
implementar (bits y programacion dinamica)


\section{Raices Primitivas en un Primo p}
Se denomina raiz primitiva $\alpha$ de p al $\alpha$ que cumple lo siguiente:
$$
    \alpha^{x_{i}} \mod p = CRR, \SP 0 \leq x_{i} \leq p-1
$$
La raiz primitiva genera el CRR (conjunto reducido de restos) del primo p. Si $\alpha$ es
raiz primitiva de p entonces tenemos las siguientes propiedades:
\begin{itemize}
    \item $\alpha^{0} \mod p = y_{0} = 1$
    \item $\alpha^{p-1} \mod p = y_{p-1} = 1$
    \item $\alpha^{x_{i}} \mod p = y_{x_{i}}, \SP x_{i} = \{1,2,...,p-3,p-2\}$
    \item $y_{x_{i}} = \{2,3,...,p-1\}$ (sin order particular)
\end{itemize}

\subsection{Problema del Logaritmo Discreto}
La formula es: $x = \log_{\alpha} \beta \mod p$, si $\alpha, \beta$ y p son conocidos
\end{document}
